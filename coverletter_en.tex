%!TEX TS-program = xelatex
%!TEX encoding = UTF-8 Unicode
% Awesome CV LaTeX Template for Cover Letter
%
% This template has been downloaded from:
% https://github.com/posquit0/Awesome-CV
%
% Authors:
% Claud D. Park <posquit0.bj@gmail.com>
% Lars Richter <mail@ayeks.de>
%
% Template license:
% CC BY-SA 4.0 (https://creativecommons.org/licenses/by-sa/4.0/)
%


%-------------------------------------------------------------------------------
% CONFIGURATIONS
%-------------------------------------------------------------------------------
% A4 paper size by default, use 'letterpaper' for US letter
\documentclass[11pt, a4paper]{awesome-cv}
\usepackage[english]{babel}
% Configure page margins with geometry
\geometry{left=1.0cm, top=.5cm, right=1.0cm, bottom=0.8cm, footskip=.5cm}

% Specify the location of the included fonts
\fontdir[fonts/]

% Color for highlights
% Awesome Colors: awesome-emerald, awesome-skyblue, awesome-red, awesome-pink, awesome-orange
%                 awesome-nephritis, awesome-concrete, awesome-darknight
%\colorlet{awesome}{awesome-red}
% Uncomment if you would like to specify your own color
% \definecolor{awesome}{HTML}{CA63A8}
\definecolor{awesome}{HTML}{007c90}
\definecolor{awesome-accent}{HTML}{fe5722}
\definecolor{awesome-shadow}{HTML}{00acc2}

% Colors for text
% Uncomment if you would like to specify your own color
% \definecolor{darktext}{HTML}{414141}
% \definecolor{text}{HTML}{333333}
% \definecolor{graytext}{HTML}{5D5D5D}
% \definecolor{lighttext}{HTML}{999999}

% Set false if you don't want to highlight section with awesome color
\setbool{acvSectionColorHighlight}{false}

% If you would like to change the social information separator from a pipe (|) to something else
\renewcommand{\acvHeaderSocialSep}{\quad}


%-------------------------------------------------------------------------------
%	PERSONAL INFORMATION
%	Comment any of the lines below if they are not required
%-------------------------------------------------------------------------------
% Available options: circle|rectangle,edge/noedge,left/right
\photo[circle]{examples/profile.jpg}
\name{Piero}{Macaluso}
% \position{Software Architect{\enskip\cdotp\enskip}Security Expert}
\position{Master of Science Computer Engineering Student at Politecnico di Torino}
\address{Via Galileo Galilei, 154 - Sanremo (IM) 18038 - Italy | Corso Lione, 24 - Torino (TO) 10141 - Italy}

\mobile{(+39) 334-7150020}
\email{info@pieromacaluso.com}
\homepage{www.pieromacaluso.com}
\github{pieromacaluso}
\linkedin{pieromacaluso}
% \gitlab{gitlab-id}
% \stackoverflow{SO-id}{SO-name}
% \twitter{@twit}
\skype{piero.macaluso@outlook.com}
% \reddit{reddit-id}
% \medium{madium-id}
% \googlescholar{googlescholar-id}{name-to-display}
%% \firstname and \lastname will be used
% \googlescholar{googlescholar-id}{}
% 4
% \extrainfo{extra informations}

%-------------------------------------------------------------------------------
%	LETTER INFORMATION
%	All of the below lines must be filled out
%-------------------------------------------------------------------------------
% The company being applied to
\recipient
  {Collegio Universitario Renato Einaudi di Torino}
  {Via Maria Vittoria 39\\10123 Torino - Italia}
% The date on the letter, default is the date of compilation
\letterdate{\today}
% The title of the letter
\lettertitle{Call for competition for a prize dedicated to a personal project in academic, educational, non-profit or professional}
% How the letter is opened
\letteropening{To Collegio Einaudi and McKinsey \& Company.}
% How the letter is closed
\letterclosing{Yours faithfully,}
% Any enclosures with the letter
\letterenclosure[Allegati]{Curriculum Vitae, Essay progetto}


%-------------------------------------------------------------------------------
\begin{document}

% Print the header with above personal informations
% Give optional argument to change alignment(C: center, L: left, R: right)
\makecvheader[L]

% Print the footer with 3 arguments(<left>, <center>, <right>)
% Leave any of these blank if they are not needed
%\makecvfooter
%  {\today}
%  {Piero Macaluso~~~·~~~Cover Letter}
%  {}

% Print the title with above letter informations
\makelettertitle

%-------------------------------------------------------------------------------
%	LETTER CONTENT
%-------------------------------------------------------------------------------
\begin{cvletter}

  
With this letter I would like to express my keen interest in participating at the competition in question by presenting the project of my master thesis which I am developing at Eurecom, graduate school and research centre located in Sophia Antipolis (Biot, France) at the forefront in the digital sciences, in particular data science and machine learning. 
I am currently completing this project, the conclusion of my master in Computer Engineering at Politecnico di Torino, where I achieved my bachelor degree with the highest marks: a great satisfaction after years of constant effort and commitment, but above all of curiosity and passion.
These two components have always been the propulsive forces that allowed me to achieve results despite the difficulties and stress that studies lead to face. I am continually looking for a fertile ground in which to apply the knowledge and tools I acquired to develop new skills.
  
During my master, I deepened many innovative aspects of my professional area, being particularly impressed by the universe of Machine Learning and Artificial Intelligence, areas in which computer science and mathematics mix themselves to emulate the chemistry and deep mechanics of the human mind, without the aim of replacing humanity, but in order to expand its potentialities, enriching it.
Precisely for this reason, I decided to invest a semester of my studies in exploring this innovative research field, accepting the challenging thesis proposal of Pietro Michiardi, the head professor of the department of Data Science at Eurecom, in collaboration with Elena Baralis, professor at Politecnico di Torino.
The main project of the thesis consists in the application of Deep Reinforcement Learning algorithms to teach a small robot how to drive from scratch on a track, using only its experiences. Reinforcement Learning is a branch of Machine Learning that takes care of resolving problems of decision-making through which an agent learns to improve his choices in performing a given task based on the reward it receives from the external environment. This research field can have critical applications in numerous and different areas, not only related to engineering. 
This period has allowed me to grow professionally and academically by making myself aware of the satisfactions and difficulties of experimental research. However, it was also a fundamental step in my career because I lived for six months in an international environment full of students from all over the world.

In 2017 I had the opportunity to work as intern in a software company, thanks to which I discovered the organization and dynamics of a company. I collaborated with some fellow interns and the team of the company in updating and testing one of their main products, acquiring new skills through the technique of pair programming.
Besides, during my studies, there were opportunities to develop personal and academic projects, often in teams. In most of these projects, I played coordinating roles that allowed me to develop skills such as leadership, team management, conflict management, time management and above all, listening. I believe that the key to success in team projects is listening, understanding the needs and proposals of each component. This fact allows you to put yourself in different points of view, discovering facets of a problem that the individual would not be able to find out for himself.

Living almost all of my university career in the merit residence "Renato Einaudi" in Turin undoubtedly represents a fundamental part of my education. I met a lot of brilliant people from Italy and beyond, thanks to whom I improved my way of interfacing with others by learning to live together, managing relationships with a considerable number of people.

Winning the grant of the call you offer in collaboration with McKinsey \& Company would be an excellent development opportunity for my research project and, more generally, for the completion of my academic studies and the start of my professional life.
Thank you for your time. I am looking forward to hearing positive news from you.

\end{cvletter}


%-------------------------------------------------------------------------------
% Print the signature and enclosures with above letter informations
\makeletterclosing

\end{document}
