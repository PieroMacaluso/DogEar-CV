%-------------------------------------------------------------------------------
%	SECTION TITLE
%-------------------------------------------------------------------------------
\cvsection{Personal Experiences and Activities}{
%-------------------------------------------------------------------------------
%	CONTENT
%-------------------------------------------------------------------------------
\vspace{-2.5pt}
\begin{cventries}
%---------------------------------------------------------
  \cventry
    {A chord archive for passionate musicians} % Affiliation/role
    {Picopod} % Organization/group
    {Sanremo, Italy} % Location
    {Jan. 2012 - Now} % Date(s)
    {
      \begin{cvitems} % Description(s) of experience/contributions/knowledge
        \item {\textbf{Description}: Picopod is a website implemented using Wordpress which aims to store an archive of music chords for all musicians. My brother and I created this website in our childhood to conjugate our shared passions: computer sciences and music. Nowadays, the site has an average of more than 1000 unique visitors each day.}
        \item {\textbf{Technologies}: Wordpress, HTML5, CSS5, PHP, MySQL, Javascript}
      \end{cvitems}
    }{\href{https://www.picopod.it}{\faLink}}

%---------------------------------------------------------
  \cventry
  {A system to count and track people position in a room using Wi-Fi devices} % Affiliation/role
  {Room Monitor} % Organization/group
  {Turin, Italy} % Location
  {Apr. 2018 - Now} % Date(s)
  {
   \begin{cvitems} % Description(s) of experience/contributions/knowledge
     \item {\textbf{Description}: This team project aims to build up a software capable of monitoring students in a classroom and collecting information about them (total number and position) using the technologies learned during the course "System Programming". In this scenario, we exploited \textit{probe request} messages sent by Wi-Fi devices, gathering them using a set of ESP32 boards (2 or more) to triangulate student position.}
      \item {\textbf{Technologies}: C, C++, Qt}
   \end{cvitems}
  }{}

  %---------------------------------------------------------
  \cventry
  {Managing bookings and schedule of a school walking bus system} % Affiliation/role
  {Pedibus} % Organization/group
  {Turin, Italy} % Location
  {Mar. 2019 - Now} % Date(s)
  {
   \begin{cvitems} % Description(s) of experience/contributions/knowledge
     \item {\textbf{Description}: This team project, developed during the course "Web Applications", aims to build up a full-stack web application to manage the service of a walking bus. It provides an easy interface for parents to book everyday rides for their children with a security notification system, but also a scheduling management system for the supporting team.}
      \item {\textbf{Technologies}: Java, Spring, Typescript, Angular, Javascript, RxJS, WebSocket, REST API, Docker}
   \end{cvitems}
  }{}

%---------------------------------------------------------
  \cventry
    {Exam Timetabling Problem Solver} % Affiliation/role
    {Timetabling Algorithm} % Organization/group
    {Turin, Italy} % Location
    {Sep. 2017 - Feb. 2018} % Date(s)
    {
      \begin{cvitems} % Description(s) of experience/contributions/knowledge
        \item {\textbf{Description}: This team project aims to propose a solution approach for extensive Exam Timetabling Problem (ETP) by exploiting one (or more) heuristic and meta-heuristic algorithms presented during the course \textit{Optimization Methods and Algorithms}. In our particular case, we used a variant of Simulated Annealing.}
        \item {\textbf{Technologies}: Java}
      \end{cvitems}
    }{\href{https://github.com/pieromacaluso/ETPsolver_OMAMZ_group09}{\faGithub}}
\end{cventries}
}
