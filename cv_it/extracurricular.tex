%-------------------------------------------------------------------------------
%	SECTION TITLE
%-------------------------------------------------------------------------------
\cvsection{Esperienze Personali e Attività}{
%-------------------------------------------------------------------------------
%	CONTENT
%-------------------------------------------------------------------------------
\vspace{-2.5pt}
\begin{cventries}
%---------------------------------------------------------
  \cventry
    {Un archivio di accordi per musicisti appassionati} % Affiliation/role
    {Picopod} % Organization/group
    {Sanremo, Italia} % Location
    {Gen. 2012 - Ora} % Date(s)
    {
      \begin{cvitems} % Description(s) of experience/contributions/knowledge
        \item {\textbf{Descrizione}: Picopod è un sito implementato con Wordpress che si presenta come un archivio di accordi musicali per tutti i tipi di musicisti. Mio fratello e io abbiamo creato questo sito nella nostra infanzia per coniugare le nostre passioni condivise: l'informatica e la musica. Ad oggi il sito ha una media di più di 1000 visitatori unici ogni giorno.}
        \item {\textbf{Tecnologie}: Wordpress, HTML5, CSS5, PHP, MySQL, Javascript}
      \end{cvitems}
    }{\href{https://www.picopod.it}{\faLink}}

%---------------------------------------------------------
  \cventry
  {Sistema per conteggio e monitoraggio posizione persone in un'aula utilizzando dispositivi Wi-Fi} % Affiliation/role
  {Room Monitor} % Organization/group
  {Torino, Italia} % Location
  {Apr. 2018 - Ora} % Date(s)
  {
   \begin{cvitems} % Description(s) of experience/contributions/knowledge
     \item {\textbf{Descrizione}: Questo progetto in team ha l'obiettivo di sviluppare un software capace di monitorare gli studenti in un'aula e collezionare informazioni sugli stessi (numero totale e posizione) utilizzando le tecnologia imparate durante il corso "Programmazione di Sistema". In questo scenario, abbiamo sfruttato i messaggi \textit{probe request} inviari dai dispositivi Wi-Fi, catturandoli con un gruppo di schede ESP32 (2 o più) per triangolare le posizioni degli studenti.}
      \item {\textbf{Tecnologie}: C, C++, Qt}
   \end{cvitems}
  }{}

  %---------------------------------------------------------
  \cventry
  {Gestione prenotazioni e programmazione delle corse di un sistema di pedibus scolastico} % Affiliation/role
  {Pedibus} % Organization/group
  {Torino, Italia} % Location
  {Mar. 2019 - Ora} % Date(s)
  {
   \begin{cvitems} % Description(s) of experience/contributions/knowledge
     \item {\textbf{Descrizione}: Questo progetto in team, sviluppato durante il corso "Applicazioni Internet", ha l'obiettivo di implementare un'applicazione web full-stack per gestire un servizio di pedibus scolastico. Offre un'interfaccia facile per permettere ai genitori di prenotare le corse per i loro bambini con un sistema di notifiche di sicurezza, ma anche per gestire le disponibilità del team di supporto al servizio.}
      \item {\textbf{Tecnologie}: Java, Spring, Typescript, Angular, Javascript, RxJS, WebSocket, REST API, Docker}
   \end{cvitems}
  }{}

%---------------------------------------------------------
  \cventry
    {Risolutore per il problema di timetabling esami} % Affiliation/role
    {Algoritmo Timetabling} % Organization/group
    {Torino, Italia} % Location
    {Set. 2017 - Feb. 2018} % Date(s)
    {
      \begin{cvitems} % Description(s) of experience/contributions/knowledge
        \item {\textbf{Descrizione}: Questo progetto in team si occupa di proporre un approccio alla soluzione di un esteso Exam Timetabling Problem (ETP) sfruttando uno (o più) algoritmi euristici o meta-euristici presentati nel corso \textit{Optimization Methods and Algorithms}. Nel nostro caso particolare abbiamo usato una variante del Simulated Annealing.}
        \item {\textbf{Tecnologie}: Java}
      \end{cvitems}
    }{\href{https://github.com/pieromacaluso/ETPsolver_OMAMZ_group09}{\faGithub}}
\end{cventries}
}
