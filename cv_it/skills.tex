%-------------------------------------------------------------------------------
%	SECTION TITLE
%-------------------------------------------------------------------------------
\cvsection{Competenze}{
\vspace{-4pt}
\cvsubsection{Professionali}

%-------------------------------------------------------------------------------
%	CONTENT
%-------------------------------------------------------------------------------
\begin{cvskills}

%---------------------------------------------------------
  \cvskill
    {Linguaggi di Programmazione} % Category
    {C, C++, Java, Python, Javascript, TypeScript, PHP, MatLab, LaTex, Bash, Assembly x86} % Skills

%---------------------------------------------------------
  \cvskill
    {Machine Learning Framework} % Category
    {NumPy, PyTorch, Scikit-Learn, Matplotlib, OpenAI Gym, TensorboardX } % Skills

%---------------------------------------------------------

  \cvskill
    {Applicazioni Web} % Category
    {Spring, REST API, HTML5, CSS3, Angular, Wordpress, Docker} % Skills

%---------------------------------------------------------
\cvskill
    {Software/IDEs} % Category
    {CLion, PyCharm, PhpStorm, WebStorm, VisualStudio Code, Matlab, TexStudio, Android Studio} % Skills

%---------------------------------------------------------
\cvskill
    {Sistemi Operativi} % Category
    {Windows, Linux (Ubuntu), Mac OSX, Android, iOS} % Skills

%---------------------------------------------------------
\cvskill
    {Database} % Category
    {SQL, MySQL, OracleSQL, Hadoop, Spark} % Skills

%---------------------------------------------------------
\cvskill
    {Altro} % Category
    {Adobe Photoshop, Adobe Illustrator, Markdown, Latex, Office Suite - ECDL Core (2011) and Advanced (2013)} % Skills

%---------------------------------------------------------

\end{cvskills}
\vspace{-5mm}
\begin{cvparagraph}
  Ho sviluppato le mie competenze informatiche durante i miei studi triennali grazie ai progetti e i laboratori didattici. Un'esperienza importante in questo senso è rappresentata dal tirocinio svolto con Feedback Italia dove ho scoperto come funziona un ambiente aziendale e come lavorare in maniera proficua in un gruppo. Il semestre che ho investito nella tesi rappresenta la più preziosa esperienza della mia carriera finora perchè ho avuto modo di fare ricerca in un ambito completamente nuovo e innovativo. Mi sono ritrovato in un ambiente davvero stimolante e familiare, pieno di brillanti professori e dottorandi, ma soprattutto pieno di idee originali sulle ultime tecnologie informatiche.
\end{cvparagraph}
\vspace{-3mm}
\cvsubsection{Organizzative e Comunicative}
\begin{cvskills}

%---------------------------------------------------------
\cvskill
{Scrittura e Esposizione} % Category
{Interazioni con persone differenti in contesti professionali, esperienze nella scrittura di report e tesi} % Skills

%---------------------------------------------------------
\cvskill
{Gestione} % Category
{Autonomia, Flessibilità, Adattabilità, Pianificazione, Raggiungimento degli obiettivi, Spirito imprenditoriale e iniziativa} % Skills

%---------------------------------------------------------
\cvskill
{Sviluppo Personale} % Category
{Fiducia in se stessi, resistenza allo stress, precisione e attenzione ai dettagli, apprendimento continuo, risoluzione dei problemi} % Skills

%---------------------------------------------------------
\cvskill
{Sviluppo Team} % Category
{Leadership, organizzazione team, divisione dei compiti, pianificazione delle tappe, risoluzione dei conflitti} % Skills


%---------------------------------------------------------

\cvskill
{Patenti e certificazioni} % Category
{Patente di guida italiana A1 e B (2013), FIRST Cambridge Grade C (2013)} % Skills

%---------------------------------------------------------

\cvskill
{Lingue} % Category
{Italiano (Madrelingua), Inglese (Livello B2 - Comprensione, Dialogo, Ascolto), Francese (Conoscenza Base)} % Skills

%---------------------------------------------------------


\end{cvskills}
\vspace{-5mm}
\begin{cvparagraph}
  Durante la mia adolescenza ho partecipato con impegno nello staff di iniziative di volontariato come educatore, dove ho acquisito capacità di moderazione funzionali nella gestione di eventi con grandi gruppi di persone, ma anche capacità di confrontare idee e comprendere differenti punti di vista. Inoltre, durante le mie esperienze professionali e di ricerca, ho acquisito abilità comunicative attraverso il continuo contatto e confronto con clienti e colleghi. Vivere nel collegio universitario di merito "Renato Einaudi" mi ha permesso di sviluppare ulteriormente le mie capacità comunicative venendo a contatto quotidianamente con un gran numero di persone e studenti.
\end{cvparagraph}
}