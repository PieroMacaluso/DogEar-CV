%!TEX TS-program = xelatex
%!TEX encoding = UTF-8 Unicode
% Awesome CV LaTeX Template for Cover Letter
%
% This template has been downloaded from:
% https://github.com/posquit0/Awesome-CV
%
% Authors:
% Claud D. Park <posquit0.bj@gmail.com>
% Lars Richter <mail@ayeks.de>
%
% Template license:
% CC BY-SA 4.0 (https://creativecommons.org/licenses/by-sa/4.0/)
%


%-------------------------------------------------------------------------------
% CONFIGURATIONS
%-------------------------------------------------------------------------------
% A4 paper size by default, use 'letterpaper' for US letter
\documentclass[11pt, a4paper]{awesome-cv}
\usepackage[italian]{babel}

% Configure page margins with geometry
\geometry{left=1.0cm, top=.5cm, right=1.0cm, bottom=0.8cm, footskip=.5cm}

% Specify the location of the included fonts
\fontdir[fonts/]

% Color for highlights
% Awesome Colors: awesome-emerald, awesome-skyblue, awesome-red, awesome-pink, awesome-orange
%                 awesome-nephritis, awesome-concrete, awesome-darknight
%\colorlet{awesome}{awesome-red}
% Uncomment if you would like to specify your own color
% \definecolor{awesome}{HTML}{CA63A8}
\definecolor{awesome}{HTML}{007c90}
\definecolor{awesome-accent}{HTML}{fe5722}
\definecolor{awesome-shadow}{HTML}{00acc2}

% Colors for text
% Uncomment if you would like to specify your own color
% \definecolor{darktext}{HTML}{414141}
% \definecolor{text}{HTML}{333333}
% \definecolor{graytext}{HTML}{5D5D5D}
% \definecolor{lighttext}{HTML}{999999}

% Set false if you don't want to highlight section with awesome color
\setbool{acvSectionColorHighlight}{false}

% If you would like to change the social information separator from a pipe (|) to something else
\renewcommand{\acvHeaderSocialSep}{\quad}


%-------------------------------------------------------------------------------
%	PERSONAL INFORMATION
%	Comment any of the lines below if they are not required
%-------------------------------------------------------------------------------
% Available options: circle|rectangle,edge/noedge,left/right
\photo[circle]{examples/profile.jpg}
\name{Piero}{Macaluso}
% \position{Software Architect{\enskip\cdotp\enskip}Security Expert}
\position{Studente magistrale in Ingegneria Informatica presso il Politecnico di Torino}
\address{Via Galileo Galilei, 154 - Sanremo (IM) 18038 - Italia | Corso Lione, 24 - Torino (TO) 10141 - Italia}

\mobile{(+39) 334-7150020}
\email{info@pieromacaluso.com}
\homepage{www.pieromacaluso.com}
\github{pieromacaluso}
\linkedin{pieromacaluso}
% \gitlab{gitlab-id}
% \stackoverflow{SO-id}{SO-name}
% \twitter{@twit}
\skype{piero.macaluso@outlook.com}
% \reddit{reddit-id}
% \medium{madium-id}
% \googlescholar{googlescholar-id}{name-to-display}
%% \firstname and \lastname will be used
% \googlescholar{googlescholar-id}{}
% 4
% \extrainfo{extra informations}

%-------------------------------------------------------------------------------
%	LETTER INFORMATION
%	All of the below lines must be filled out
%-------------------------------------------------------------------------------
% The company being applied to
\recipient
  {Collegio Universitario Renato Einaudi di Torino}
  {Via Maria Vittoria 39\\10123 Torino - Italia}
% The date on the letter, default is the date of compilation
\letterdate{\today}
% The title of the letter
\lettertitle{Bando di concorso per premio dedicato a un progetto personale in ambito accademico, formativo, no profit o professionale}
% How the letter is opened
\letteropening{Al Collegio Einaudi e a McKinsey \& Company.}
% How the letter is closed
\letterclosing{Cordiali saluti,}
% Any enclosures with the letter
\letterenclosure[Allegati]{Curriculum Vitae, Essay progetto}


%-------------------------------------------------------------------------------
\begin{document}

% Print the header with above personal informations
% Give optional argument to change alignment(C: center, L: left, R: right)
\makecvheader[L]

% Print the footer with 3 arguments(<left>, <center>, <right>)
% Leave any of these blank if they are not needed
%\makecvfooter
%  {\today}
%  {Piero Macaluso~~~·~~~Cover Letter}
%  {}

% Print the title with above letter informations
\makelettertitle

%-------------------------------------------------------------------------------
%	LETTER CONTENT
%-------------------------------------------------------------------------------
\begin{cvletter}

Con questa lettera vorrei esprimere il mio vivo interesse a partecipare al bando di concorso in oggetto presentando il mio progetto di tesi sperimentale che sto sviluppando presso Eurecom, graduate school e centro di ricerca situato in Sophia Antipolis (Biot, Francia) all'avanguardia nelle scienze digitali e in particolare in ambiti quali data science e machine learning. 
Attualmente sto portando a termine questo progetto, conclusione del mio percorso di studi magistrale in Ingegneria Informatica presso il Politecnico di Torino, dove ho avuto modo di conseguire il titolo triennale con il massimo dei voti: una grande soddisfazione dopo anni di sforzi e impegno costante, ma soprattutto di curiosità e passione.
Proprio queste due componenti sono state da sempre le forze propulsive che mi hanno permesso di raggiungere risultati nonostante le difficoltà e lo stress che gli studi portano ad affrontare. Sono alla costante ricerca di un terreno fertile in cui applicare le conoscenze e gli strumenti acquisiti per poter sviluppare nuove competenze. 

Durante il mio percorso di studi magistrali ho approfondito molti aspetti innovativi del mio settore rimanendo particolarmente colpito dall'universo del Machine Learning e dell'Intelligenza Artificiale, ambiti in cui informatica e matematica si mescolano per cercare di emulare la chimica e le meccaniche profonde della mente umana, non con l'obiettivo di sostituire l'uomo, ma per poter espandere le sue potenzialità, arricchendolo.
Proprio per questo motivo ho deciso di investire un semestre del mio percorso di studi per poter esplorare questo ambito di ricerca innovativo, accettando la stimolante sfida racchiusa nella proposta di tesi del prof. Pietro Michiardi, capo del dipartimento di Data Science presso Eurecom, in collaborazione con la prof. Elena Baralis del Politecnico di Torino.
L'argomento principale della tesi consiste nell'applicazione di alcuni algoritmi di Deep Reinforcement Learning, un settore del Machine Learning che si occupa di risolvere problemi di decision-making attraverso cui l'agente impara a migliorare le proprie scelte nello svolgere un determinato task in base alla ricompensa che riceve dall'ambiente esterno per ogni azione che esegue: ambito che può avere applicazioni davvero importanti in numerosi e differenti ambiti, non solo legati all'ingegneria. Nel progetto in sviluppo, questi algoritmi vengono applicati per insegnare ad un piccolo robot come guidare su un tracciato partendo da zero, sfruttando solo le sue esperienze.
Questo periodo mi ha permesso di poter crescere dal punto di vista professionale e accademico facendomi conoscere le soddisfazioni e le difficoltà della ricerca sperimentale, ma è stato fondamentale anche dal punto di vista sociale, poichè ho avuto la possibilità di vivere in un ambiente internazionale con studenti provenienti da tutto il mondo.

Nel 2017 ho avuto l'opportunità di svolgere un tirocinio curriculare, grazie al quale ho scoperto l'organizzazione e le dinamiche di un'azienda. Ho collaborato con alcuni colleghi tirocinanti e il team principale della compagnia nell'aggiornamento e testing di uno dei loro prodotti principali, acquisendo competenze nuove attraverso la tecnica del pair programming.
Inoltre, durante gli  studi non sono mancate le possibilità di sviluppare progetti personali e accademici, spesso in team. Nella maggior parte di questi progetti ho potuto svolgere ruoli di coordinamento che mi hanno permesso di sviluppare competenze quali leadership, gestione del team, gestione dei conflitti, time management e soprattutto ascolto. Credo che la chiave del successo quando il progetto viene svolto da un team di persone sia ascoltare, comprendere le necessità e le proposte di ognuno. Questo permette di porsi in punti di vista differenti dal proprio, scoprendo sfaccettature di un problema che il singolo non riuscirebbe a scoprire da sè.

Parte fondamentale della mia formazione è rappresentata sicuramente dal Collegio Einaudi in cui ho avuto modo di vivere la quasi totalità del mio percorso universitario. Ho conosciuto numerose persone brillanti da tutta Italia e non solo, grazie alle quali ho migliorato il mio modo di interfacciarmi con gli altri imparando a vivere e convivere, gestendo relazioni con un numero considerevole di persone.

Vincere la borsa da voi offerta in collaborazione con McKinsey \& Company rappresenterebbe una ottima opportunità di sviluppo per il mio progetto di ricerca e, più in generale, per la conclusione della formazione accademica e l'inizio della formazione professionale.
Vi ringrazio per il tempo dedicatomi e spero di poter ricevere presto notizie positive.

\end{cvletter}


%-------------------------------------------------------------------------------
% Print the signature and enclosures with above letter informations
\makeletterclosing

\end{document}
